\Chapter{Összegzés}

A dolgozat keretein belül be sikerült bizonyítani, hogy szükség van térképadatbázisra kétdimenziós játékokhoz.
Ez a játéklogika fejlesztési idejét lecsökkentheti, mivel kész eszközöket ad a programozók kezébe. Jelenleg az adatbázismotor nem tartalmaz optimalizált algoritmusokat, viszont az az interfész változtatása nélkül pótolható. A forráskódok elérhetőek \textit{GitHub}-on, így az interneten közzétéve lehetőség nyílik arra, hogy más játékfejlesztők igénybe vegyék, vagy teszteljék a szoftvert, észrevételeket tudjanak küldeni azzal kapcsolatban.

Ahhoz, hogy a piacon versenyképes termék lehessen az adatbázismotorból, ahhoz még néhány funkciót implementálni kell, illetve optimalizált algoritmusokkal szeretném támogatni a szolgáltatásokat.

Mivel jelenleg egyetlen kész eszközt sem implementáltak az adott probléma megoldására, amely ellátná az általam kitalált és megvalósított adatbázismotor szolgáltatásait, ezért ha optimalizált formában is elérhető lesz az adatbázismotor és a hozzá tartozó lekérdezőnyelv, akkor minden általam készített kétdimenziós játékban szeretném használni, és reményeim szerint nagyobb játékfejlesztő közösség is szívesen építi majd alkalmazásait az adatbázismotorra.

Korábban már több játékfejlesztő jelezte, hogy szeretné majd használni az adatbázismotort a saját projektjeihez. Remélem a közeljövőben egy hatékony szoftvereszközt adhatok a kezükbe.

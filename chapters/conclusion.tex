\Chapter{Összegzés}

A dolgozat keretein belül sikerült bebizonyítani, hogy szükség van térképadatbázisra kétdimenziós játékokhoz, amely a játéklogikát programozók számára lecsökkentheti a fejlesztési időt azzal, hogy kész eszközöket ad számára. Jelenleg az adatbázismotor nem tartalmaz optimalizált algoritmusokat, fő szempont a létjogosultság bizonyítása volt. A forráskódok elérhetőek GitHubon, így az interneten közzétéve lehetőség nyílik arra, hogy más játékfejlesztők igénybe vegyék, vagy teszteljék a szoftvert, észrevételeiket pedig eljuttassák hozzánk. Ahhoz, hogy a piacon versenyképes termék lehessen az adatbázismotorból, ahhoz még néhány funkciót implementálni kell, illetve optimalizált algoritmusokkal szeretném támogatni a szolgáltatásokat. Mivel jelenleg egyetlen kész eszközt sem implementáltak az adott probléma megoldására, amely ellátná az általunk kitalált és megvalósított adatbázismotor szolgáltatásait, ezért ha teljesen kész lesz a motor és a hozzá tartozó lekérdezőnyelv, akkor minden általam készített kétdimenziós játékban szeretném használni, és reményeim szerint nagyobb játékfejlesztő közösség is szívesen építené az alkalmazás logikáját az adatbázismotorra.
Több játékfejlesztő jelezte már, hogy szeretné majd használni az adatbázismotort a saját saját projektjeihez, remélem a közeljövőben kész eszközt adhatunk a kezükbe.

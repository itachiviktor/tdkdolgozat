\Chapter{Bevezetés}

Számos alkalmazásban (tipikusan például a számítógépes játékokban) szükség  van kétdimenziós térképek adatainak tárolására, és az azokon lévő entitások kezelésére. Ennek a megvalósítása az egyes implementációkban sok közös elemet tartalmaz. Az általam definiált lekérdezőnyelv, és a hozzá készített adatbázis kezelő rendszer eszközt biztosít a programozók számára, hogy a térképek elemeit, és az elemek attribútumait egyszerűbben, egy lekérdezőnyelvi interfészen keresztül érhessék el. A térképen lévő elemeket entitásoknak nevezzük. A nyelvben definiálhatjuk az entitások osztályait. Az adatbázismotor a gyakori műveletekhez nyelvi szintű támogatást ad. Megfogalmazhatunk például olyan lekérdezéseket, amelyekkel entitások ütközését, vagy az azok közötti távolságokat vizsgáljuk. Adatbázisról lévén szó a fejlesztőknek nem szükséges a térképadatok perzisztens tárolási módjával külön foglalkozniuk.

Az ötlet, hogy szükség lenne egy ilyen domain specifikus lekérdezőnyelvre, játékfejlesztés közben merült fel, illetve fejlesztői fórumok olvasgatása közben, ugyanis sokan kétségbe voltak esve, hogyan is lehetne kétdimenziós játékhoz pályát generálni, és kezelni azt, ezért hosszas kutatás után a témában megállapítottuk, hogy ilyen szolgáltatás egyenlőre még nem készült, viszont igény lenne a meglétére.

A dolgozatnak nem célja, hogy piacon versenyképes szoftvert mutasson be, sokkal inkább arra helyeznénk a hangsúlyt, hogy alátámasszuk a létjogosultságát.

\begin{comment}
Le kell majd írni, hogy az elnevezések (pl.: azeroth), és a példák miért ilyenek.
\end{comment}

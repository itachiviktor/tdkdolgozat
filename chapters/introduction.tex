\Chapter{Bevezetés}

Számos alkalmazásban (tipikusan például a számítógépes játékokban) szükség  van kétdimenziós térképek adatainak tárolására, és az azokon lévő entitások kezelésére. Ennek a megvalósítása az egyes implementációkban sok közös elemet tartalmaz. Az általam definiált lekérdezőnyelv, és a hozzá készített adatbázis kezelő rendszer eszközt biztosít a programozók számára, hogy a térképek elemeit, és az elemek attribútumait egyszerűbben, egy lekérdezőnyelvi interfészen keresztül érhessék el. A térképen lévő elemeket entitásoknak nevezzük. A nyelvben definiálhatjuk az entitások osztályait. Az adatbázismotor a gyakori műveletekhez nyelvi szintű támogatást ad. Megfogalmazhatunk például olyan lekérdezéseket, amelyekkel entitások ütközését, vagy az azok közötti távolságokat vizsgáljuk. Adatbázisról lévén szó a fejlesztőknek nem szükséges a térképadatok perzisztens tárolási módjával külön foglalkozniuk.
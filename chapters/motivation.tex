\Chapter{Domain specifikus lekérdezőnyelvek}

A DSL egy olyan nyelv, amely egy adott feladatkör problémáinak megoldására lett létrehozva. Magasabb absztrakciós szinten lehet vele a problémákat megfogalmazni.
Elrejt alacsony szintű műveleteket, például változók felszabadítása.

TODO: Ide azt kellene leírni, hogy
\begin{itemize}
\item Milyen más területeken van szükség még domain specifikus nyelvekre?

Beágyazott szkript nyelvek esetén beszélhetünk DSL nyelvről, ilyenkor valamilyen specializált feladat ellátását támogató kód készíthető a segítségével. Az SAP vállalatirányítási rendszerhez is készült DSL szkript nyelv. Az SAPScript az egy eszköz az SAP rendszeren belül, amely eszközt biztosít , hogy formokat tervezzünk, nyomtassunk, stb.

Geoinformatika: A földmérés, távérzékelés, földrajz és térképészet új módszereit és lehetőségeit kutató tudomány. A térinformatikai rendszerek alkalmazása nagyon széles. A geográfiai adatok kezeléséhez is létrehoztak DSL nyelveket. Vannak olyan adatbázisok, melyek kifejezetten térkép adatok tárolására, és azok kezelésére lettek tervezve.


\item A játékfejlesztésben milyen problémák vannak, amelyekre egy adatbázismotor megoldást tud adni?

Térképadatok mellett, az üzleti logika perzisztens tárolását is lehetővé teszi.

A játékfejlesztésben az ütközés vizsgálat megoldása amennyire fontos feladat, oly annyira nehéz is. Hogyan észleljük az ütközést és mi történjen ennek hatására?
Ezen problémákkal nem kell a játékfejlesztőnek foglalkoznia, ezeket nyelvi szinten szolgáltatásként igénybe vehetik az adatbázismotortól.

Az ütközésvizsgálat mellett az útvonal tervezés sem triviális feladat. Legrövidebb út meghatározása két pont között úgy, hogy közben a térképen lévő entitásokat kikerülje.

A térkép felépítése nem a programozó feladata, ő lekérdezésekkel létrehozhatja azt.

Entitások közötti távolság kiszámítása sem az üzleti logika fejlesztőinek feladata, az adatbázismotor elvégzi ezt is helyettük.



\item Hogyan integrálható be egy ilyen szoftverkomponens/szoftvereszköz a fejlesztési folyamatba?
1. Külső (external) DSL: Szintaktikája eltér a szoftver megírásához használt programozási nyelvétől. A külső DSL szintaktikája lehet saját fejlesztésű, vagy épülhet egy szabványos formátumra, például XML-re.
Példák: reguláris kifejezések, SQL.

2. Belső (internal) DSL: A befogadó nyelv egy kis részhalmazát használja, jellemzően a programozási eszközök egy szűk körére szorítkozik, vagy a nyelvet sajátos stílusban használja.
\end{itemize}
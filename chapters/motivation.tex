\Chapter{Térképadatok kezelése}

\section{Domain specifikus nyelvek}

Domain specifikus nyelveknek (\textit{DSL - Domain Specific Language}) azokat nevezzük, amelyeket egy adott, specializált feladatkör problémáinak megoldására hoztak létre. Segítségükkel magasabb absztrakciós szinten lehet a problémákat megfogalmazni. Elrejt alacsony szintű műveleteket, mint például a szükséges modulok explicit kijelölését, az objektumok létrehozásának és lebontásának alacsony szintű műveleteit.

A terület specifikus nyelv az alábbi módokon jelenhet meg.

\begin{itemize}
\item \textit{Külső (external) DSL}: Szintaktikája eltér a szoftver megírásához használt programozási nyelvétől. A külső DSL szintaktikája lehet saját fejlesztésű, vagy épülhet egy szabványos formátumra, például XML-re. Ilyenek például a reguláris kifejezések leírásához használt nyelvek, illetve az SQL.
\item \textit{Belső (internal) DSL}: A befogadó nyelv egy kis részhalmazára épül. Jellemzően a programozási eszközök egy szűk körére szorítkozik, vagy a nyelvet sajátos stílusban használja.
\end{itemize}

% Milyen más területeken van szükség még domain specifikus nyelvekre?

A beágyazott szkript nyelvek között sok DSL nyelvet találhatunk. Ilyen esetben egy specializált feladat ellátását támogató kód készíthető a segítségével. Például az SAP vállalatirányítási rendszerhez is definiáltak  DSL szkript nyelvet. Az \textit{SAPScript} az egy eszköz az SAP rendszeren belül, amellyel formokat tervezhetünk, azokat egyszerűen nyomtathatjuk.

A terület specifikus nyelv tehát adott problémakörben felmerülő feladatokra ad hatékony megoldást.

\section{Térképadatbázisok}

A térképadatok kezelési módja jelentősen eltér a hagyományosan táblázatos formában ábrázolható adatokétól. A geoinformatikában gyakran szükség van ilyen adatok nyilvántartására, például földmérés, távérzékelés és egyéb földrajzi adatok esetén. Ahhoz, hogy ez hatékonyan menjen végbe térképadatbázisokat hoztak létre, amelyek sajátos adattípusokat és műveleteket biztosítanak.

\begin{comment}{Megnézni pár példát, hogy milyen implementációk, típusok és műveletek vannak!}
\end{comment}

% A játékfejlesztésben milyen problémák vannak, amelyekre egy adatbázismotor megoldást tud adni?

\section{Térképek szerepe a játékfejlesztésben}

A számítógépes játékok jelentős részében egy virtuálisan létrehozott világot járhatunk be.

A játékfejlesztés során rengeteg olyan probléma felmerül, amely szorosan kötődik a térképekhez. Egy ilyen például az ütközés vizsgálata. Amennyiben nem szeretnénk, hogy a kiterjedéssel rendelkező objektumaink át tudjanak menni egymáson, ellenőrizni kell, hogy ütköznek-e. Erre a játékmotorok biztosítanak bizonyos funkciókat, viszont a játék egyedi vonásai, illetve a jobb teljesítmény érdekében a fejlesztőknek ezen gyakran módosításokat kell végeznie.

Hasonló problémát jelent az útvonalak megtervezése. Ezt úgy érdemes megtenni, hogy az útvonal kikerüli a térképünkön lévő többi objektumot, miközben törekszik arra az algoritmus, hogy ez minél rövidebb legyen. Ez szintén nem triviális feladat, viszont gyakran szükség van rá.

\section{Fejlesztést segítő szoftvereszközök}

A szoftver fejlesztése során számos olyan probléma felmerül, amelyeket nem szeretnénk minden esetben újra megoldani. Térképek esetében ilyen például az említett ütközésvizsgálat és útvonaltervezés.

A fejlesztők munkáját többféleképpen segíthetjük. Az algoritmusok közreadása mellett létrehozhatunk olyan függvénykönyvtárat, illetve adatbázis implementációt, amely a gyakran előforduló feladatokat segít megoldani egy jól definiált interfész megadásával. Ez esetben egy olyan terület specifikus lekérdezőnyelv adhatja a megoldást, amely egy kellőképpen általános adatmodellt, és műveleti részt definiál a térképadatok kezeléséhez. A dolgozatban egy ilyen nyelv bemutatására kerül sor. A fejlesztőknek így elegendő a nyelv felépítésével és használati módjával megismerkedni. A lekérdezések értelmezésének és végrehajtásának részleteit már nem kell ismerniük.

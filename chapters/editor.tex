\Chapter{Grafikus szerkesztőeszköz}

\section{Az editor funkciói}

\begin{itemize}
\item Admin felület: Ennek a grafikus felületnek az a célja, hogy az
adatbázissal kapcsolatos műveleteket grafikus felületen kezelhessük, nem pedig konzolos parancsok beírásával.
\item Térkép szerkesztő: Grafikus felületen összerakhatjuk a játékban szereplő térképet, és a háttérben ezeket a lekérdezéseket az editor adja ki.
\item Debug eszköz: Beírhatunk lekérdezéseket is a szerkesztőbe, és a lekérdezés eredményét rögtön láthatjuk is a térképen. Ha elemek kijelölése volt a lekérdezésben, akkor a térképen piros keretet rak az eredményhalmazban szereplő entitások köré.
	
\end{itemize}

\section{Implementáció}

A grafikus szerkesztőeszközt is Java nyelven implementáltam. Itt a szerkesztő az adatbázismotort könyvtárként foglalja magába. Az editor az egy egyszerű Swinges alkalmazásként lett megvalósítva. 